% SampleReportTemplate.tex
\documentclass[11pt,a4paper]{report}

% Packages
\usepackage[utf8]{inputenc}
\usepackage[T1]{fontenc}
\usepackage[english]{babel}
\usepackage{geometry}
\usepackage{graphicx}
\usepackage{booktabs}
\usepackage{caption}
\usepackage{braket}
\usepackage{amsmath,amssymb}
\usepackage{siunitx}        % nice units and uncertainties
\usepackage{hyperref}
\usepackage{fancyhdr}
\usepackage{setspace}
\usepackage{titling}       % for custom titlepage fields
\usepackage{enumitem}
\renewcommand\thesection{\arabic{section}}
\renewcommand\thesubsection{\thesection.\arabic{subsection}}
% Page geometry and spacing
\geometry{
  left=30mm,
  right=25mm,
  top=25mm,
  bottom=30mm,
  headsep=10pt
}
\onehalfspacing

% Header / Footer
\pagestyle{fancy}
\fancyhf{}
\fancyhead[L]{\small Quantum Laboratory}
\fancyhead[R]{\small \LabTitle}
\fancyfoot[C]{\thepage}
\renewcommand{\headrulewidth}{0.4pt}
\renewcommand{\footrulewidth}{0pt}

% Metadata commands (edit these)
\newcommand{\LabTitle}{Fundamentals of Quantum Optics}
\newcommand{\ContactName}{Prof. Frank Setzpfandt}
\newcommand{\ContactAffil}{Friedrich-Schiller-Universit\"at Jena\\Abbe School of Photonics}

\newcommand{\ContactPhone}{+49 (0) 3641 947569}
\newcommand{\ContactEmail}{f.setzpfandt@uni-jena.de}



\newcommand{\TAName}{Sabine H\"aussler}
\newcommand{\DateOfLab}{December 4, 2025}
\newcommand{\DateOfFinalReport}{December 11, 2025}

% Titlepage
% \pretitle{\begin{titlepage}\begin{center}\vspace*{10mm}}
% \posttitle{\end{center}
% \end{titlepage}}
\setlength{\droptitle}{-5mm}

\begin{document}

% ---------- Custom Title Page ----------
\begin{titlepage}
  \begin{flushleft}
    \textbf{Contact person:}\\[2mm]
    \ContactName\\
    \ContactAffil\\
    Phone: \ContactPhone, \\
    e-mail: \texttt{\ContactEmail}\\[10mm]

  \end{flushleft}

  \vspace{20mm}

  \begin{center}
    {\LARGE\bfseries \LabTitle \par}
    \vspace{15mm}

    \begin{tabular}{ll}
      \textbf{Lab Title}:    & \LabTitle \\[2mm]
      \textbf{Group number}: & 5 \\[2mm]
      \textbf{Student name}: & Ishaan Chaturvedi \\[2mm]
      \textbf{Name of Teaching Assistant}: & \TAName \\[2mm]
      \textbf{Date of Lab}: & \DateOfLab \\[2mm]
      \textbf{Date of Report Submission}: & \DateOfFinalReport \\
    \end{tabular}
  \end{center}

  \vfill
  \begin{center}
    {\large \textbf{Quantum Laboratory}}\\[2mm]
    \textit{Abbe School of Photonics, Friedrich-Schiller-Universit\"at Jena}
  \end{center}
\end{titlepage}
% ---------- End Title Page ----------

\pagenumbering{roman}
\tableofcontents
\cleardoublepage
\pagenumbering{arabic}

% ================= Sections =================

\section{Introduction}
Secure Communication has always been an important 
\section{Theory}
In conventional cryptography, the information is shared using public keys for encryption
and private keys for decryption. However, the security of these methods 
relies on the computational difficulty of a computer. For example, RSA method
utilises the fact that it is much harder to factorise a 
number into its prime factors than to multiply two prime numbers together. However,
with the advent of quantum computers, we have algorithms like Shor's algorithm
which can efficiently factorise large numbers, thus breaking the security of RSA.
So there is a need for cryptographic methods that are secure against such
attacks. This is where Quantum Key Distribution (QKD) comes into play.  Instead of 
relying on computational difficulty, QKD used the principles of quantum
mechanics to ensure that the key distribution is secure. In this experiment we will 
discuss the BB84 protocol, which is one of the first and most well-known QKD
protocols. 
\subsection{BB84 Quantum Key Distribution Protocol}
BB84 protocol was proposed by Charles Bennett and Gilles Brassard in 1984. It utilises
the properties of quantum mechanics to securely distribute a cryptographic key.
Once we share a secret key, we can use it to encrypt and decrypt messages which can now 
be transmitted using a classical channel. In order to share a key using BB84   
protocol, we will take two communicating people, conventionally called Alice 
and Bob, who want to share a secret key. In order to share the key with BB84 protocal 
they need to follow the following steps:
\begin{enumerate}
    \item \textbf{Preparation and Transmission:} Alice randomly selects a sequence of bits
Here randomness is crucial for the success of the protocol. She also randomly chooses
a basis (rectilinear or diagonal) for each bit. She will then encode each randomly chosen bit in the 
polarisation state of the photon according to the randomly chosen basis and sends the photons to BOB.
    \item \textbf{Bob'Measurement:} Upon receiving the photons, Bob randomly chooses a basis (rectilinear or diagonal) 
    for measuring each incoming photon. He records the measurement results and store them in the
    form of bits. Note that the bit encoding scheme for polarisation states is 
    the same as Alice's encoding scheme. However, since Bob is choosing his measurement
    basis randomly, there will be instances where his chosen basis does not match
    Alice's encoding basis. Thus to make sense of the bits, they
    will need to communicate over a classical channel.
    \item \textbf{Classical communication}: Now Alice and Bob communicate over
    a classical channel. Bob tells Alice which basis he used for each measurement,
    but he will not reveal the measurement results. Alice then reveals to Bob
    which measurements were done in the correct basis. They will discard all thebibliography
    bits where the measurement basis did not match. The remaining bits constitute
    the potential secret key.
    \item \textbf{Error Rate Estimation and eavesdropper detection:} In order to check the security of the 
     communication channel, Alice 



\end{enumerate}

\subsection{Quantum States and Bases}
Information is encoded using four quantum states: $\ket{0}$, $\ket{1}$ in the rectilinear basis, and $\ket{+} = \frac{1}{\sqrt{2}}(\ket{0} + \ket{1})$, $\ket{-} = \frac{1}{\sqrt{2}}(\ket{0} - \ket{1})$ in the diagonal basis. An eavesdropper attempting to intercept the quantum states will introduce detectable errors due to quantum mechanics principles.

\section{Experimental Realization}

\section{Measurement Results and Analysis}

\section{Conclusions}
















% ================ Bibliography ==================

\phantomsection
\addcontentsline{toc}{chapter}{Bibliography}
\begin{thebibliography}{9}
\bibitem{nasa:speedofsound}
 NASA, \emph{Speed of Sound}, Online resource. % Provide URL or further info
\bibitem{nordling:physics}
 C.~Nordling and J.~Ostermann, \emph{Physics Handbook for Science and Engineering}, 8th ed., Studentlitteratur, 2006.
\bibitem{heatcap:gases}
 Heat Capacity Ratio for Gases, Online resource.
\bibitem{wikipedia}
 Wikipedia, Online encyclopedia.
\end{thebibliography}

% Optional: if you use biblatex:
% \printbibliography

\end{document}
